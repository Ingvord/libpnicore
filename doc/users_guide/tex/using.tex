%%% description of how to use the library 
\libpnicore\ provides a {\tt pkg-config} file. In the case of a system wide
installation this file is most probably allready at the right place in the file
system. One can easily check this with 
\begin{minted}{bash}
>> pkg-config --libs --cflags pnicore 
\end{minted}
for a system wide installation you should get something like this
\begin{verbatim}
-lpniio -lhdf5 -lz -lboost_filesystem -lpnicore -lboost_program_options\ 
-lboost_regex -lboost_system 
\end{verbatim}
For installation locations which are not in the default paths of your system you
may get some additional {\tt -I} and {\tt -L} output pointing to the directories
where the header files and the library binaries are installed.
If {\tt pkg-config} complains that it cannot find a package named {\tt pnicore}
then you most probably have to set {\tt PKG\_CONFIG\_PATH} to the location where 
the {\tt pkg-config} file of your \libpnicore\ installation has been installed. 

%%%============================================================================
\section{From the command line}

If a single simple program should be compiled the following approach is
suggested 
\begin{verbatim}
$> g++ -std=c++11 -O3 -oprogram program.cpp\
       $(pkg-config --cflags --libs pnicore)
\end{verbatim}
Please recognize the {\tt -std=c++11} option. \libpnicore\ requires a state of
the art compiler with full support for C++11.

%%%============================================================================
\section{From within a Makefile}

If {\tt make} should be used to build the code add the following lines to your
{\tt Makefile}

\begin{minted}{make}
CPPFLAGS=-g -O3 -std=c++11 -fno-deduce-init-list -Wall -Wextra -pedantic \
		 $(shell pkg-config --cflags pnicore) 
LDFLAGS=$(shell pkg-config --libs pnicore)
\end{minted}
This will set the appropriate compiler and linker options for the build.

%%%============================================================================
\section{With Scons}

If you use {\tt SCons} for building the code add the following to your {\tt
SConstruct} file
\begin{minted}{python}
env = Environment()
env.AppendUnique(CXXFLAGS=["-std=c++11","-pedantic","-Wall","-Wextra"])
env.ParseConfig('pkg-config --cflags --libs pnicore')
\end{minted}
The {\tt ParseConfig} method of a {\tt SCons} environment is able to parse the
output of {\tt pkg-config} and add the flags to the environments configuration.

%%%============================================================================
\section{With CMake}

For \cmake\ package files are provides which makes integrating \libpnicore\ into
other projects using \cmake\ rather easy.

