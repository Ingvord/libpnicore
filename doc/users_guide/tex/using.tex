%%% description of how to use the library 
\section{Include files}

To use \libpnicore\ in code the appropriate header files must be included. In
the simplest case just use the \texttt{core.hpp} file 
\begin{minted}{cpp}
#include <pni/core.hpp>
\end{minted}
which pulls in all the other header files required to work with \libpnicore.
Alternatively, the header files for the different components provided by
\libpnicore\ can be used
\begin{minted}{cpp}
#include <pni/core/algorithms.hpp>    //basic algorithms
#include <pni/core/arrays.hpp>        //multidimensional arrays
#include <pni/core/benchmark.hpp>     //benchmark classes
#include <pni/core/configuration.hpp> //program configuration facilities
#include <pni/core/error.hpp>         //exception management
#include <pni/core/type_erasures.hpp> //type erasures
#include <pni/core/types.hpp>         //fundamental data types
#include <pni/core/utilities.hpp>     //general purpose utilities
\end{minted}
All classes provided by \libpnicore\ reside within the \texttt{pni::core}
namespace. If you do not want to give the namespace explicitly for every type
and function use
\begin{minted}{cpp}
using namespace pni::core;
\end{minted}
after including the required header files.

\section{Building and linking}
\libpnicore\ provides a \texttt{pkg-config} file. In the case of a system wide
installation this file is most probably allready at the right place in the file
system. One can easily check this with 
\begin{minted}{bash}
>> pkg-config --libs --cflags pnicore 
\end{minted}
for a system wide installation you should get something like this
\begin{verbatim}
-lpniio -lhdf5 -lz -lboost_filesystem -lpnicore -lboost_program_options\ 
-lboost_regex -lboost_system 
\end{verbatim}
For installation locations which are not in the default paths of your system you
may get some additional \texttt{-I} and \texttt{-L} output pointing to the directories
where the header files and the library binaries are installed.
If \texttt{pkg-config} complains that it cannot find a package named \texttt{pnicore}
then you most probably have to set \texttt{PKG\_CONFIG\_PATH} to the location where 
the \texttt{pkg-config} file of your \libpnicore\ installation has been installed. 

%%%============================================================================
\subsection{From the command line}

If a single simple program should be compiled the following approach is
suggested 
\begin{verbatim}
$> g++ -std=c++11 -O3 -oprogram program.cpp\
       $(pkg-config --cflags --libs pnicore)
\end{verbatim}
Please recognize the \texttt{-std=c++11} option. \libpnicore\ requires a state of
the art compiler with full support for C++11.

%%%============================================================================
\subsection{From within a Makefile}

If \texttt{make} should be used to build the code add the following lines to your
\texttt{Makefile}

\begin{minted}{make}
CPPFLAGS=-g -O3 -std=c++11 -fno-deduce-init-list -Wall -Wextra -pedantic \
		 $(shell pkg-config --cflags pnicore) 
LDFLAGS=$(shell pkg-config --libs pnicore)
\end{minted}
This will set the appropriate compiler and linker options for the build.

%%%============================================================================
\subsection{With Scons}

If you use \texttt{SCons} for building the code add the following to your 
\texttt{SConstruct} file
\begin{minted}{python}
env = Environment()
env.AppendUnique(CXXFLAGS=["-std=c++11","-pedantic","-Wall","-Wextra"])
env.ParseConfig('pkg-config --cflags --libs pnicore')
\end{minted}
The \texttt{ParseConfig} method of a \texttt{SCons} environment is able to parse the
output of \texttt{pkg-config} and add the flags to the environments configuration.

%%%============================================================================
\subsection{With CMake}

Currently no \cmake-package files are installed with the library. However, due
to \cmake s \texttt{pkg-config} support autoconfiguration can still be done. In one
of the toplevel \texttt{CMakeLists.txt} files use
\begin{minted}{cmake}
pkg_search_module(PNICORE REQUIRED pnicore)
\end{minted}
to load the configuration for \libpnicore\ from \texttt{pkg-config}. Furthermore we
have to add the header file and library installation paths to the configuration
\begin{minted}{cmake}
link_directories(${PNICORE_LIBRARY_DIRS} ${HDF5_LIBRARY_DIRS})
include_directories(${PNICORE_INCLUDE_DIRS} ${HDF5_INCLUDE_DIRS})
\end{minted}
Finally the library can easily be added to the build target with 
\begin{minted}{cmake}
target_link_libraries(mytarget ${PNICORE_LIBRARIES})
\end{minted}

