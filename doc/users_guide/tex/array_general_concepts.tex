
\libpnicore\ currently provides two array templates 
%%%----------------------------------------------------------------------------
\begin{center}
\begin{tabular}{m{0.1\linewidth}p{0.7\linewidth}}
\sdarray &  
static arrays whose complete shape and type is know at compile time\\
\darray & 
a dynamic array for which the type must be determined at compile
time but its shape (and thus size) can be configured at runtime \\
\end{tabular}
\end{center}
%%%----------------------------------------------------------------------------
In addition to this two basic templates there are several utility classes and
templates like
%%%----------------------------------------------------------------------------
\begin{center}
\begin{tabular}{m{0.15\linewidth}p{0.7\linewidth}}
\arrayview &  a template providing a particular view on an array (see
Sec.~\ref{sec:array:array_slicing}) \\
\numarray &  an adapter template for arrays providing arithmetic operations
(see Sec.~\ref{sec:array:numeric_arrays}) \\
\arrayerasure & a type erasure that can be used with any instance of an array
template (see Sec.~\ref{sec:array:type_erasure})
\end{tabular}
\end{center}
%%%----------------------------------------------------------------------------

\subsection{Common array interface}

All array templates share the following common implicit interface
\begin{minted}{cpp}
template<typename T> class array_interface
{
public:
//==================public types===========================================
typedef ... value_type;             // element type
typedef ... shared_ptr;             // a shared pointer to the array
typedef ... unique_ptr;             // a unique pointer to the array
typedef ... iterator;               // iterator
typedef ... const_iterator;         // constant iterator
typedef ... reverse_iterator;       // reverse iterator type
typedef ... const_reverse_iterator; //const reverse iterator type
typedef ... array_type;             // array type
typedef ... storage_type;           // type of the storage
typedef ... view_type;              // type of the view
typedef ... map_type;               // type of the index map
//================public members===========================================
static const type_id_t type_id = type_id_map<value_type>::type_id;

//==================constructor and destructor=============================
array_interface();                         //need default constructor
array_interface(const array_interface &a); //copy constructor

//=======================assignment operator===============================
array_type &operator=(const array_type &a); //copy assignment operator

//======================methods for multiindex data access=================
//if all arguments are integer numbers return a reference to the
template<typename ...ITYPES> 
typename array_view_selector<array_type,ITYPES...>::reftype 
operator()(ITYPES ...indices);

template<typename ...ITYPES> 
typename array_view_selector<array_type,ITYPES...>::viewtype 
operator()(ITYPES ...indices);

template<template<typename ...> class CTYPE,typename ...OTS> 
typename array_view_selector<array_type,typename CTYPE::value_type>::reftype
operator()(const CTYPE<OTS...> &index);

template<template<typename ...> class CTYPE,typename ...OTS> 
typename array_view_selector<array_type,typename CTYPE::value_type>::viewtype
operator()(const CTYPE<OTS...> &index) const;

//================methods for linear data access===========================
//operators for linear data access - these operators are assumed not to
//throw any exception
value_type &operator[](size_t i);
value_type  operator[](size_t i) const;

//methods for linear data access with index checking - these methods are
//assumed to throw IndexError if the index exceeds the size of the array
value_type &at(size_t i);        //reference to the element at index i
value_type  at(size_t i) const;  //value of the element at index i

//insert an element at index i
void        insert(size_t pos,const value_type &v);

value_type &front();       //reference to the first element
value_type  front() const; //value of the first element
value_type &back();        //reference to the last element
value_type  back() const;  //value of the last element

//get a const. reference to the storage object
const storage_type &storage() const;

//===============methods for array inquery==================================
size_t size() const; //return number of elements of type T
size_t rank() const; //number of dimensions

//return a container of type CTYPE with the number of elements along each
//dimension
template<typename CTYPE> shape() const;

const map_type &map() const; //return reference to the index map

//===================iterator related methods==============================
iterator begin();              //iterator on the first element
iterator end();                //iterator on the last+1 element
const_iterator begin() const;  //const iterator on the first element
const_iterator end()   const;  //const iterator on the last+1 element

reverse_iterator rbegin();     //reverse iterator on the last element
reverse_iterator rend();       //reverse iterator on the 0-1 element

//const reverse iterator on the last element
const_reverse_iterator rbegin() const; 
//const reverse iterator on the 0-1 element
const_reverse_iterator rend()   const;
};
\end{minted}

This interface is quite similar to the interface of {\tt std::vector} or other
STL containers. Thus, all instances of array templates can be used with STL
algorithms. There is one important note concerning array construction and
assignment: all arrays have to provide a default constructor and a copy
constructor as well as a copy assignment operator. However, static and dynamic
arrays and some of their derived types also provide a move constructor and
assignment operator. Hence, they can be returned by functions without additional
copy operations.
