%%% Dealing with type erasures

As powerful templates are the tedious they can become in certain situations. 
For this purpose several type erasures are available. 
\begin{anfxnote}[layout=inline]{Introduction type erasures}
Need to be a bit more explicit here about what type reasures are
and when to use them. Maybe add a reference to one of the online resources on 
type erasures.
\end{anfxnote}
The following erasures are available 
\begin{description}
\item[{\tt value}]
can be used to store a single scalar value of a POD type. 
\item[{\tt value\_ref}]
stores a reference to a single scalar POD type.
\item[{\tt array}] 
holds an arbitrary array type.
\end{description}
To use type erasures include the {\tt /pni/core/type\_erasures.hpp} at the top of
your source file.
The default way of how to create a {\tt value} type erasure would be the
following
\inputminted[fontsize=\small,firstnumber=35,firstline=35,lastline=37]{cpp}
{../examples/type_erasure1.cpp}
where a value is passed to the constructor.
To retrieve the value stored in a {\tt value} use the {\tt as} template method
as shown in the next snippet
\inputminted[fontsize=\small,firstnumber=40,firstline=40,lastline=41]{cpp}
{../examples/type_erasure1.cpp}
The {\tt as} template function performs srict type checking. If the template
parameter does not exactly match the data type stored in the erasure a {\tt
type\_error} exception will be thrown.

Though type erasures can be default constructed without any value stored this
would leave their internal memory unallocated. An attempt to retrieve a value
from such a default constructed type erasure instance will throw a {\tt
memory\_not\_allocated} exception. For a clean construction of type erasures 
use the {\tt value::create} template function as show in the next snippet
\inputminted[fontsize=\small,firstnumber=44,firstline=44,lastline=46]{cpp}
{../examples/type_erasure1.cpp}
The {\tt value::create} template function stores a default object of the
requested type. However, it is not important which type to use (maybe we can
omit this part). With the next assginment of a new data value new memory is
allocated. 

The {\tt value} type erasures not only stores an instance of a particular
type it also takes over ownership. In some cases one may wants to use a type
erasures as a reference.
