%%% Dealing with type erasures

As powerful templates are the tedious they can become in certain situations. 
For this purpose several type erasures are available [REFERENCE TO TYPE
ERASURES]. 
The following erasures are available 
\begin{description}
\item[{\tt value}]
can be used to store a single scalar value of a POD type. 
\item[{\tt value\_ref}]
stores a reference to a single scalar POD type.
\item[{\tt array}] 
holds an arbitrary array type.
\end{description}
To use type erasures include the {\tt /pni/core/type\_erasures.hpp} at the top of
your source file.
The default way of how to create a {\tt value} type erasure would be the
following
\inputminted[fontsize=\small,linenos,firstnumber=35,firstline=35,lastline=37]{cpp}
{../examples/type_erasure1.cpp}
where a value is passed to the constructor.
To retrieve the value stored in a {\tt value} use the {\tt as} template method
as shown in the next snippet
\inputminted[fontsize=\small,linenos,firstnumber=40,firstline=40,lastline=41]{cpp}
{../examples/type_erasure1.cpp}
Due to the internal structure of a type erasure construction without a target
value leaves the type 


