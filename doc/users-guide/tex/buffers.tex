%%
%% Buffers chapters for the users guide (latex file) 
%%
%% (c) Copyright 2011 DESY, Eugen Wintersberger <eugen.wintersberger@desy.de>
%%
%% This file is part of libpniutils.
%%
%% libpniutils is free software: you can redistribute it and/or modify
%% it under the terms of the GNU General Public License as published by
%% the Free Software Foundation, either version 2 of the License, or
%% (at your option) any later version.
%%
%% libpniutils is distributed in the hope that it will be useful,
%% but WITHOUT ANY WARRANTY; without even the implied warranty of
%% MERCHANTABILITY or FITNESS FOR A PARTICULAR PURPOSE.  See the
%% GNU General Public License for more details.
%%
%% You should have received a copy of the GNU General Public License
%% along with libpniutils.  If not, see <http://www.gnu.org/licenses/>.
%%************************************************************************
%%
%% Chapter about Buffer objects for the users guide (latex file)
%%
%% Created on: Sep 15, 2011
%%     Author: Eugen Wintersberger <eugen.wintersberger@desy.de>
%%
%%

%%%some text about buffers
\section{Buffer objects}
Buffers play an important role within the library. Usually you do not need 
to care about them. However, for library developers it is important 
to know how they work.

A buffer represents a number of elements of a particular size linearly 
stored in memory. Thus is behaves somehow like an array or one of the 
container templates provided by the STL. However, it follows its own 
allocation and reallocation policy which will be described here in more detail.  

The aim of a buffer is to allocate memory for a number of objects of equal 
size. Therefore, the amount of memory allocated depends on two parameters
\begin{enumerate}
  \item the {\em element size } which is the size of each element in bytes
  \item the {\em size } of the buffer which is the number of elements in the
  buffer.
\end{enumerate}
The total amount of memory which must be allocated to hold this data is 
than given with
\begin{align}
n_{\mathrm{tot}} = \mathrm{element size}\times\mathrm{size}.
\end{align}

All buffers should be derived from the {\tt BufferObject} base class which 
holds the minimal interface each buffer object should implement. 
The base class itself only provides simple managment functions for the
size of the allocated memory. However, it already implies some of the 
policies that each buffer object should follow. 
There are three situations a buffer object must face
\begin{enumerate}
  \item instantiation
  \item allocation
  \item assignment
  \item reallocation
\end{enumerate} 

\subsection{Instantiation of buffer objects}

Buffers allow the allocation of memory to be postponed to a later point in 
time after their creation. This of coarse has some implications. 
It is necessary to keep track of a buffers state in every method that 
can effect the allocated memory. 

The easy way to allocate a buffer is the following
\begin{verbatim}
Buffer<UInt16> b(1024);
\end{verbatim}
using this constructor memory is allocated right at the time the object is 
created. 
If memory allocation should be done latter you can do the following
\begin{verbatim}
Buffer<UInt16> b1,b2;

b1.setSize(1024);
b1.allocate();

//alternatively
b2.allocate(1024);
\end{verbatim}
It is important to note that once a buffer is allocated a call to {\tt
allocate()} will throw an exception. The same is valid for a call to 
{\tt setSize()} after the allocation of a buffer. 
It is necessary to explicitly free a buffer before reallocating it 
or reseting its size:
\begin{verbatim}
Buffer<UInt16> b(1024);

b.allocate(100);  //exception  is thrown
b.setSize(100);   //exception is thrown

//this will work
b.free()
b.setSize(100);
b.allocate();
\end{verbatim}
This sounds maybe a bit unsusual but should force people to really think 
about memory allocation. 
The problem with such an approach is that the implementation of the 
copy constructor and the assignment operator (see below) is a bit tricky. 
For the copy constructor one can at least assume that one of the objects
is designed from scratch. 




The template {\tt Buffer<T>} is a concrete implementation 
