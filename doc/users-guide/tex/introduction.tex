%%
%% Users guide introduction chapter (latex file) 
%%
%% (c) Copyright 2011 DESY, Eugen Wintersberger <eugen.wintersberger@desy.de>
%%
%% This file is part of libpniutils.
%%
%% libpniutils is free software: you can redistribute it and/or modify
%% it under the terms of the GNU General Public License as published by
%% the Free Software Foundation, either version 2 of the License, or
%% (at your option) any later version.
%%
%% libpniutils is distributed in the hope that it will be useful,
%% but WITHOUT ANY WARRANTY; without even the implied warranty of
%% MERCHANTABILITY or FITNESS FOR A PARTICULAR PURPOSE.  See the
%% GNU General Public License for more details.
%%
%% You should have received a copy of the GNU General Public License
%% along with libpniutils.  If not, see <http://www.gnu.org/licenses/>.
%%************************************************************************
%%
%% Users guide introduction chapter (latex file)
%%
%% Created on: Aug 31, 2011
%%     Author: Eugen Wintersberger <eugen.wintersberger@desy.de>
%%
%%

%%% introduction to the PNI utilities library

The PNI\footnote{PNI stands for Photon, Neutron, and Ion.} utilities library
(libpniutils) provides, as its name already indicates, a couple of utility 
classes that should make  life easier for developers writting 
software that should run on PNI-facilities. Thus, this guide is the first 
thing new developers should read before using the library. 

C++ does not provide any sophisticated data structures to manipulate large 
amounts of data. This is in particular a problem for users that do not 
want to play with the programming language but rather need to finish their 
work within reasonable time. 
Python became quite prominent in the recent times within the scientific 
community. One of the reasons was the vast amount of modules available and 
a small amount of 3rd party modules like numpy or scipy which make 
working with multidimensional numeric arrays as easy as in Fortran. 
